
\newacronym{tfg}{TFG}{Trabajo Fin de Grado}
\newacronym{etse}{ETSE-UV}{Escuela Técnica Superior de Ingeniería}
\newacronym{uveg}{UV}{Universitat de València}
\newacronym{tfm}{TFM}{Trabajo Fin de Máster}
\newacronym{html}{HTML}{HyperText Markup Language}
\newacronym{sql}{SQL}{Structured Query Language}
\newacronym{rest}{REST}{Representational State Transfer}
\newacronym[\glsshortpluralkey=SSGGBBDD]{sgbd}{SGBD}{Sistema de Gestión de Base de Datos}
\newacronym{nist}{NIST}{National Institute of Standards and Technology}
\newacronym{aws}{AWS}{Amazon Web Services}
\newacronym{gcp}{GCP}{Google Cloud Platform}
\newacronym{cdn}{CDN}{Content Delivery Network}
\newacronym{k8s}{K8S}{Kubernetes}
\newacronym{ine}{INE}{Instituto Nacional de Estadística}
\newacronym{iso}{ISO}{International Organization for Standardization}
\newacronym{ec2}{EC2}{Elastic Compute Cloud}
\newacronym{yaml}{YAML}{YAML Ain't Markup Language}
\newacronym{s3}{S3}{Simple Storage Service}
\newacronym{rds}{RDS}{Relational Database Service}
\newacronym{eks}{EKS}{Elastic Kubernetes Service}
\newacronym{vm}{VM}{Virtual Machines}
\newacronym{seo}{SEO}{Optimización para Motores de Búsqueda}
\newacronym{appservices}{AppServices}{Azure App Services}
\newacronym{cloudfunctions}{CloudFunctions}{Google Cloud Functions}
\newacronym{blobstorage}{BlobStorage}{Azure Blob Storage}
\newacronym{computeengine}{ComputeEngine}{Google Compute Engine}
\newacronym{cloudstorage}{CloudStorage}{Google Cloud Storage}
\newacronym{aeat}{AEAT}{Agencia Estatal de Administración Tributaria}
\newacronym{iaas}{IaaS}{Infraestructure as a Service}
\newacronym{rrss}{RR.SS}{Redes Sociales}
\newacronym{spa}{SPA}{Single Page Application}
\newacronym{lcp}{LCP}{Largest Contentful Paint}
\newacronym{paas}{PaaS}{Platform as a Service}
\newdualentry{html5}{HTML5}{HyperText Markup Language 5.0}{Ultima versión de \Gls{html}}
\newacronym{dto}{DTO}{Data Transfer Object}
\newacronym{tls}{TLS}{Transport Layer Security}
\newacronym{mvc}{MVC}{Modelo Vista Controlador}
\newacronym{jwt}{JWT}{JSON Web Token}
\newacronym{cors}{CORS}{Cross-Origin Resource Sharing}
\newacronym{xpath}{XPath}{XML Path Language}
\newacronym{https}{HTTPS}{HyperText Transfer Protocol Secure}
\newacronym{xml}{XML}{eXtensible Markup Language}
\newacronym{css}{CSS}{Cascading Style Sheets}
\newacronym{json}{JSON}{JavaScript Object Notation}
\newacronym{dns}{DNS}{Domain Name System}
\newacronym{tcp}{TCP}{Transmission Control Protocol}
\newacronym{udp}{UDP}{User Datagram Protocol}
\newacronym{ftp}{FTP}{File Transfer Protocol}
\newacronym{ssh}{SSH}{Secure Shell}
\newacronym{ssl}{SSL}{Secure Sockets Layer}
\newacronym{e2e}{E2E}{End to End}
\newacronym{PERT}{PERT}{Program Evaluation and Review Technique}
\newacronym{ci}{CI}{Continuous Integration}
\newacronym{twcam}{TWCAM}{Tecnologías Web, Aplicaciones Moviles y Servicios}
\newacronym{aemet}{AEMET}{Agencia Estatal de Meteorología}
\newacronym{xss}{XSS}{Cross-Site Scripting}
\newacronym{sqlinjection}{SQL Injection}{Inyección de SQL}
\newacronym[\glsshortpluralkey=APIs]{API}{API}{Application Programming Interface}
\newacronym{csp}{CSP}{Content Security Policy}
\newglossaryentry{Gantt}{
    name={Gantt},
    description={Un diagrama de Gantt es una herramienta de gestión de proyectos que se utiliza para representar visualmente el cronograma de un proyecto. Cada tarea se representa como una barra horizontal, cuyo tamaño y posición dependen de su duración y fecha de inicio. El diagrama permite visualizar la relación entre las diferentes tareas, identificar los plazos y controlar el progreso de un proyecto. Es utilizado ampliamente en la planificación de proyectos para asegurar que las tareas se completen dentro del tiempo estimado.},
    symbol={Diagrama de Gantt}
  }

  \newglossaryentry{pwa}{
    name={PWA},
    description={Una aplicación web progresiva (PWA) es un tipo de aplicación que combina lo mejor de las aplicaciones web y móviles. Se caracteriza por ser rápida, confiable y atractiva, permitiendo a los usuarios interactuar con ella incluso sin conexión a Internet. Las PWAs utilizan tecnologías web modernas para ofrecer una experiencia similar a la de una aplicación nativa, incluyendo notificaciones push, acceso a la cámara y almacenamiento local.}
    symbol={Progressive Web App}
  }

  \newglossaryentry{secret}{
name={Secret},
description={Objeto de Kubernetes que almacena información sensible, como contraseñas, tokens OAuth o claves SSH, de forma codificada en base64 y desacoplada del código}
}

\newglossaryentry{configmap}{
name={ConfigMap},
description={Objeto de Kubernetes que permite almacenar datos de configuración no confidenciales como pares clave-valor, para ser consumidos por pods o contenedores en tiempo de ejecución}
}
\newglossaryentry{dockerfile}{name= \it Dockerfile, description={Archivo de texto que contiene instrucciones para crear una imagen de Docker. Define el entorno y las dependencias necesarias para ejecutar una aplicación.}}
\newglossaryentry{dockerhub}{name= \it Docker Hub, description={Servicio de registro de imágenes de Docker que permite almacenar y compartir imágenes de contenedores.}}
\newglossaryentry{middleware}{
	name={\it Middleware},
	description={Software que se encuentra entre el sistema operativo y las aplicaciones que funcionan sobre él}
}

\newglossaryentry{sprint}{
    name={sprint},
    plural={sprints},
    description={Periodo de tiempo corto, típicamente de una a cuatro semanas, durante el cual un equipo ágil desarrolla y entrega un incremento del producto que es funcional y revisable.}
}


\newglossaryentry{openapi}{
    name={OpenAPI},
    description={Especificación para construir APIs RESTful, que define un formato estándar para describir los endpoints, parámetros, respuestas y otros aspectos de una API. Permite la generación automática de documentación y clientes para interactuar con la API.}
}
\newglossaryentry{pert}{
    name={Program Evaluation and Review Technique},
    description={Técnica de Revisión y Evaluación de Programas (Program Evaluation and Review Technique), una herramienta de gestión de proyectos que se utiliza para planificar y coordinar tareas dentro de un proyecto. Utiliza estimaciones de tiempos optimistas, más probables y pesimistas para calcular el tiempo esperado para completar una tarea.}
}

\newglossaryentry{distribucionbeta}{
    name={distribución Beta},
    description={Distribución de probabilidad utilizada en el análisis PERT para modelar incertidumbres en tiempos de ejecución de tareas, considerando los valores optimista, pesimista y más probable para generar una estimación ponderada.}
}

\newglossaryentry{planningpoker}{
    name={Planning Poker},
    description={Técnica de estimación de tareas utilizada en metodologías ágiles, donde los miembros del equipo asignan puntos de esfuerzo a las tareas mediante consenso, utilizando un conjunto de tarjetas con valores predefinidos.}
}

% Glosario de términos específicos
\newglossaryentry{frontend}{
    name=Frontend,
    description={Parte de una aplicación web que interactúa directamente con el usuario y controla la interfaz de usuario y la experiencia visual}
}
\newglossaryentry{serverless}{
    name=Serverless,
    description={Arquitectura de computación en la que los desarrolladores crean y ejecutan aplicaciones sin tener que gestionar los servidores. El proveedor de la nube es responsable de la infraestructura y del escalado, lo que permite a los desarrolladores concentrarse en la lógica de la aplicación. Aunque el término "serverless" implica que no hay servidores, en realidad los servidores están siendo gestionados por el proveedor de la nube, y los desarrolladores solo pagan por el uso de recursos cuando se ejecutan sus funciones.}
}

\newglossaryentry{fullstack}{
    name=full-stack,
    description={Hace referencia a un desarrollador o un sistema que trabaja con todas las capas de desarrollo de una aplicación, desde el front-end (interfaz de usuario) hasta el back-end (servidores y bases de datos)}
}

\newglossaryentry{backend}{
    name=Backend,
    description={Parte de una aplicación web que maneja la lógica de negocio, la gestión de datos y la comunicación con el frontend}
}

\newglossaryentry{framework}{
    name=Framework,
    plural=Frameworks,
    description={Conjunto de herramientas y librerías que proporciona una estructura de desarrollo para simplificar la construcción y el mantenimiento de aplicaciones}
}
\newglossaryentry{azure}{
    name=Azure,
    description={Microsoft Azure, una plataforma de servicios en la nube de Microsoft.}
}
\newglossaryentry{NoSQL}{
    name=NoSQL,
    description={Modelo de base de datos que no utiliza el esquema tradicional de tablas y permite el almacenamiento de datos en formatos como documentos, pares clave-valor, y grafos, ideal para grandes volúmenes de datos no estructurados}
}

\newglossaryentry{redis}{
    name=Redis,
    description={Modelo de base de datos en memoria de tipo clave-valor, utilizado para almacenamiento en caché y manejo de datos temporales}
}
\newglossaryentry{typescript}{
    name=TypeScript,
    description={Un superset de JavaScript que añade tipado estático y otras características a JavaScript.}
}


\newglossaryentry{SQL}{
    name=SQL,
    description={Lenguaje estándar para la gestión y manipulación de bases de datos relacionales que utilizan un esquema basado en tablas}
}

\newglossaryentry{contenedorizacion}{
    name=Contenerización,
    description={Proceso de empaquetar una aplicación y sus dependencias en un contenedor para asegurar su ejecución en cualquier entorno}
}

\newglossaryentry{orquestacion}{
    name=Orquestación,
    description={Proceso de gestionar, automatizar y coordinar múltiples contenedores, generalmente en un entorno de computación distribuida}
}

\newglossaryentry{sharding}{
    name=Sharding,
    description={Proceso de dividir una base de datos en partes más pequeñas, distribuyéndola en múltiples servidores para mejorar la eficiencia y escalabilidad}
}

\newglossaryentry{microservicios}{
    name=Microservicios,
    description={Arquitectura que organiza una aplicación en servicios independientes y pequeños que se comunican entre sí, facilitando la escalabilidad y el mantenimiento}
}

\newglossaryentry{caching}{
    name=Caching,
    description={Proceso de almacenar datos temporalmente en memoria rápida para reducir el tiempo de acceso en consultas repetidas}
}

\newglossaryentry{persistencia}{
    name=Persistencia,
    description={Capacidad de una base de datos de mantener los datos almacenados después de que se cierre la aplicación}
}

\newglossaryentry{DOM}{
    name=DOM (Document Object Model),
    description={Representación estructural del contenido de un documento web, permitiendo su manipulación por scripts como JavaScript}
}


% Definir glosarios
\newglossaryentry{meta}{
    name=Meta,
    description={Una compañía de tecnología que desarrolla productos como Facebook, Instagram, WhatsApp y otras aplicaciones relacionadas con la realidad virtual y aumentada.}
}

\newglossaryentry{javascript}{
    name=JavaScript,
    description={Un lenguaje de programación de alto nivel, interpretado y orientado a objetos, utilizado principalmente para crear interactividad en sitios web.}
}


\newglossaryentry{angular}{
    name=Angular,
    description={Un framework de desarrollo de aplicaciones web basado en TypeScript, mantenido por Google, que facilita la creación de aplicaciones de una sola página.}
}

\newglossaryentry{react}{
    name=React,
    description={Una biblioteca de JavaScript desarrollada por Meta para la construcción de interfaces de usuario basadas en componentes.}
}

\newglossaryentry{nodejs}{
    name=Node.js,
    description={Un entorno de ejecución de JavaScript en el lado del servidor que permite ejecutar código JavaScript fuera del navegador.}
}

\newglossaryentry{docker}{
    name=Docker,
    description={Una plataforma para desarrollar, enviar y ejecutar aplicaciones dentro de contenedores, lo que permite un entorno aislado y replicable.}
}

\newglossaryentry{kubernetes}{
    name=Kubernetes,
    description={Un sistema de orquestación de contenedores que automatiza el despliegue, la escalabilidad y la gestión de aplicaciones en contenedores.}
}

\newglossaryentry{mongodb}{
    name=MongoDB,
    description={Una base de datos NoSQL orientada a documentos que permite almacenar datos en formato BSON, ideal para aplicaciones con esquemas flexibles.}
}

\newglossaryentry{postgresql}{
    name=PostgreSQL,
    description={Una base de datos relacional de código abierto, conocida por su fiabilidad, robustez y características avanzadas de gestión de datos.}
}
\newglossaryentry{ingress}{
    name={Ingress Controller},
    description={Componente de Kubernetes que gestiona el acceso externo a los servicios del clúster, normalmente HTTP/HTTPS, mediante reglas de enrutamiento}
}
\newglossaryentry{mysql}{
    name=MySQL,
    description={Un sistema de gestión de bases de datos relacional de código abierto, ampliamente utilizado en aplicaciones web por su simplicidad y eficiencia.}
}
\newglossaryentry{pv}{
    name={PersistentVolume},
    description={Recurso de almacenamiento en Kubernetes que representa un volumen físico o lógico existente, gestionado por el administrador del clúster}
}

\newglossaryentry{pvc}{
    name={PersistentVolumeClaim},
    description={Solicitud de almacenamiento persistente en Kubernetes, que un contenedor realiza para vincularse a un volumen disponible}
}

\newglossaryentry{token}{
    name={token},
    description={Archivo o valor secreto utilizado para autenticación o autorización segura entre servicios}
}

\newglossaryentry{microservicio}{
    name={microservicio},
    plural={microservicios},
    description={Estilo de arquitectura donde la funcionalidad se divide en pequeños servicios independientes que se comunican entre sí, normalmente a través de APIs}
}

\newglossaryentry{memcached}{
    name=Memcached,
    description={Un sistema de almacenamiento en caché distribuido basado en memoria, utilizado para acelerar el acceso a datos en aplicaciones web.}
}

\newglossaryentry{cassandra}{
    name=Cassandra,
    description={Una base de datos NoSQL distribuida diseñada para manejar grandes volúmenes de datos, ideal para aplicaciones que requieren alta disponibilidad.}
}

\newglossaryentry{springboot}{
    name=Spring Boot,
    description={Un marco de trabajo basado en Java que facilita la creación de aplicaciones backend escalables, incluyendo microservicios.}
}

\newglossaryentry{express}{
    name=Express,
    description={Un framework minimalista para Node.js que facilita la creación de aplicaciones web y APIs REST.}
}

