% Contenidos del capítulo
%Las secciones presentadas son orientativas y no representan necesariamente la organización que debe tener este capítulo.

\section{Requisitos}
En este capítulo se presentan los requisitos funcionales y no funcionales del sistema propuesto para la gestión de una casa rural, detallando las funcionalidades necesarias desde el punto de vista de los usuarios y del propietario. También se incluyen la estimación de costes, la planificación temporal del proyecto y los recursos utilizados para su desarrollo.
\subsection{Descripción del sistema}

El sistema desarrollado permitirá gestionar de manera integral una casa rural, proporcionando una plataforma intuitiva y funcional tanto para los gestores como para los usuarios. Ofrecerá un menú principal con información relevante sobre las instalaciones y actividades disponibles, así como un módulo de reservas en línea que facilitará la interacción entre los huéspedes y el gestor.

La plataforma contará con un sistema de recomendaciones que, de manera automática, ofrecerá sugerencias de sitios turísticos y eventos en la zona, personalizando dichas recomendaciones según las preferencias del usuario. Asimismo, se integrarán mapas interactivos para mostrar rutas y facilitar el acceso a la casa rural, así como información útil sobre la gastronomía y actividades al aire libre.

Además, el sistema brindará la posibilidad de contactar al propietario a través de un formulario de contacto y permitirá a los usuarios dejar reseñas sobre su experiencia, enriqueciendo así la interacción con futuros huéspedes. También se ofrecerá un pronóstico del clima para ayudar a los usuarios a planificar sus actividades, y un feed actualizado de las redes sociales para estar al tanto de las novedades de la casa rural.


\subsection{Requisitos funcionales}
\begin{itemize}

    \item \textbf{RF1:} La plataforma debe contar con un módulo que muestre el clima actual y el pronóstico del tiempo para los días seleccionados en la reserva, específicamente para la zona de la casa rural.
    \item \textbf{RF2:} La plataforma debe ofrecer una sección con sugerencias de actividades al aire libre, como senderismo y ciclismo, que incluya descripción de las actividades, mapas interactivos y niveles de dificultad.
    \item \textbf{RF3:} La plataforma debe incluir un calendario visual que muestre la disponibilidad de la casa rural.
    \item \textbf{RF4:} La plataforma debe incluir un apartado para visualizar reseñas que han aportado clientes de la casa rural.
    \item \textbf{RF5:} La plataforma debe tener un apartado para visualizar los feeds de las publicaciones de redes sociales de la casa rural.
    \item \textbf{RF6:} La plataforma debe contener recomendaciones sobre sitios, actividades y eventos cercanos a la localización de la casa rural.
    \item \textbf{RF7:} La plataforma debe mostrar un menú principal con información general sobre la casa rural, incluyendo imágenes, vídeos y una descripción de las instalaciones.
    \item \textbf{RF8:} La plataforma debe proporcionar información sobre la gastronomía y otros aspectos turísticos del pueblo.
    \item \textbf{RF9:} La plataforma debe incluir un mapa interactivo que facilite la llegada a la casa rural, utilizando servicios de mapas integrados de Google Maps.
    \item \textbf{RF10:} La plataforma debe tener un apartado para realizar reservas, permitiendo a los usuarios registrados seleccionar una fecha en línea.
    \item \textbf{RF11:} La base de datos debe almacenar información sobre las reservas y los usuarios que las han realizado, para que esta información esté disponible en línea para todos los usuarios.
    \item \textbf{RF12:} Cada registro de reserva debe incluir el rango de fechas, el número de personas, el precio total, el estado de la reserva y el usuario que realizó la reserva.
    \item \textbf{RF13:} Cada registro de usuario debe contener información como DNI/NIE/Pasaporte, nombre, apellidos, e-mail, teléfono, dirección, fecha de nacimiento, contraseña (cifrada), fecha de registro, fecha de baja y estado de la cuenta.
    \item \textbf{RF14:} El sistema debe permitir a los usuarios registrados, con al menos una experiencia en la casa, dejar reseñas sobre su experiencia en la casa rural, y estas opiniones deben ser mostradas en la página.
    \item \textbf{RF15:} El sistema debe permitir a los usuarios contactar a la gerencia de la casa mediante un formulario que envíe un correo al gestor. Este formulario debe incluir los campos de nombre, e-mail, teléfono y mensaje.
    \item \textbf{RF16:} El sistema debe integrar las \glspl{API} de una aplicación de rutas para obtener y alimentar el menú de actividades al aire libre.
    \item \textbf{RF17:} El sistema debe implementar servicios de extracción de datos de páginas web de ayuntamientos y entidades cercanas que expongan blogs con eventos, tarjetas de recomendaciones de lugares o datos sobre climatología de forma automatizada.
    \item \textbf{RF18:} El sistema debe integrar las \glspl{API} de las redes sociales de la casa rural para obtener publicaciones, que se almacenarán en una base de datos documental. Estas publicaciones se mostrarán en la plataforma en tiempo real.
    \item \textbf{RF19:} El sistema debe utilizar los hashtags identificados en las publicaciones de redes sociales, almacenados junto a los feeds, para determinar en qué menús mostrar las publicaciones, llenando así los menús de recomendaciones, eventos, actividades e imágenes del alojamiento en la plataforma.
    \item \textbf{RF20:} El sistema debe avisar al usuario gestor por correo cuando se realice una solicitud de reserva, tras ser almacenada en el sistema.
    \item \textbf{RF21:} El sistema debe avisar al usuario cliente por correo cuando se aprueba o deniegue una solicitud de reserva.
    \item \textbf{RF22:} El sistema debe estar diseñado para tres tipos de usuarios: el gestor, el usuario cliente registrado y el usuario no registrado.
    \item \textbf{RF23:} El sistema debe implementar un sistema de autenticación y autorización para los usuarios, con roles de gestor, cliente y usuario registrado.
    \item \textbf{RF24:} El usuario cliente, además de tener acceso a todo el contenido de la plataforma, podrá realizar reservas en la casa rural. También tendrá acceso a un menú donde podrá consultar sus reservas, gestionar su cuenta y dejar una reseña tras su estancia. Su rol se mantendrá como \texttt{cliente} hasta que complete la reseña correspondiente.
    \item \textbf{RF25:} El usuario gestor, además de tener acceso a todo el contenido de la plataforma, podrá revisar las peticiones de reservas y decidir si las aprueba o las deniega.
    \item \textbf{RF26:} El usuario no registrado podrá acceder a las secciones de la plataforma que no están restringidas a usuarios registrados o al gestor.

    
\end{itemize}

\subsection{Requisitos no funcionales}
\begin{itemize}
    \item \textbf{RNF01:} La página debe cargarse en menos de 3 segundos en redes de velocidad media (50 a 200 Mbps).
    \item \textbf{RNF02:} El sistema debe proteger la información de los usuarios con protocolo \gls{tls} y medidas de seguridad en formularios de contacto.
    \item \textbf{RNF03:} El sistema debe ser escalable horizontalmente para manejar picos de tráfico y aumentos en la cantidad de datos dinámicos.
    \item \textbf{RNF04:} El sistema debe contar con un código modular y documentado para facilitar futuras actualizaciones y correcciones.
    \item \textbf{RNF05:} El sistema debe ser compatible con los principales navegadores (Chrome, Firefox, Safari, Edge) y dispositivos móviles (Android y iOS).
    \item \textbf{RNF06:} El sistema debe tener una interfaz de usuario intuitiva y accesible, con navegación sencilla entre los diferentes menús.
    \item \textbf{RNF07:} El sistema debe implementar copias de seguridad automáticas para los datos almacenados en la base de datos temporal.
    \item \textbf{RNF08:} El sistema debe tener configuraciones de monitorización continua y registro de errores para la detección y resolución rápida de problemas.
    \item \textbf{RNF09:} El sistema debe ser desplegado mediante contenedores utilizando Docker para facilitar la portabilidad y la consistencia entre entornos de desarrollo, prueba y producción.
    \item \textbf{RNF10:} El despliegue del sistema debe ser gestionado y orquestado mediante \gls{k8s} para mejorar la escalabilidad, la resiliencia y la administración de los microservicios en los servicios con más demanda.
\end{itemize}

\section{Especificaciones}\label{sec:especificaciones-sistema}
Una vez se han definido los requisitos funcionales y no funcionales del sistema, se procede a especificar el sistema a partir de ellos. En este apartado se muestran características del sistema, sin entrar en detalles de diseño e implementación, que permiten entender el problema al que nos enfrentamos pero no cómo lo vamos a afrontar (etapa de diseño).

\subsection{Descripción general}
El sistema a desarrollar es una plataforma para una casa rural que ofrece alojamiento y actividades turísticas en un entorno natural, tal y como representa la  Figura~\ref{fig:especificacion}.
\begin{figure}[h!t]
    \centering
    \includegraphics[width=\textwidth]{figs/especificacion.png}
    \caption{Arquitectura de referencia del sistema.}
    \label{fig:especificacion}
\end{figure}
La aplicación web debe mostrar información general de la casa rural, recomendaciones de sitios turísticos y eventos cercanos, información sobre la gastronomía y otros aspectos turísticos del pueblo, un formulario de contacto, un calendario de disponibilidad, un sistema de reservas, mapas y rutas, el clima y pronóstico del tiempo, actividades al aire libre, reseñas de clientes y un feed de redes sociales.

\subsection{Funcionalidades}
\begin{itemize}
    \item \textbf{Menú principal}: Muestra información general de la casa rural, incluyendo imágenes, vídeos y descripción de las instalaciones.
    \item \textbf{Recomendaciones y eventos}: Servicio que muestra recomendaciones de sitios turísticos y eventos cercanos, actualizados automáticamente.
    \item \textbf{Información turística}: Sección que muestra información sobre la gastronomía y otros aspectos turísticos del pueblo.
    \item \textbf{Formulario de contacto}: Permite a los usuarios contactar al dueño de la casa mediante un formulario que incluye campos de nombre, email, teléfono y mensaje.
    \item \textbf{Calendario de disponibilidad}: Calendario visual que muestra la disponibilidad de la casa rural.
    \item \textbf{Sistema de reservas}: Permite al usuario reservar una fecha en línea y gestiona la aprobación o denegación de la reserva, concluyendo con un correo de confirmación o rechazo al usuario.
    \item \textbf{Mapas y rutas}: Muestra mapas y rutas para facilitar la llegada a la casa rural, utilizando integración con servicios de mapas.
    \item \textbf{Clima y pronóstico}: Módulo que muestra el clima actual y el pronóstico del tiempo para los días seleccionados en la reserva en la zona de la casa rural.
    \item \textbf{Actividades}: Sección con sugerencias de actividades al aire libre como senderismo y ciclismo, con mapas interactivos y niveles de dificultad, actualizados automáticamente según la ubicación.
    \item \textbf{Reseñas de huéspedes}: Permite a los huéspedes dejar reseñas sobre su experiencia en la casa rural y muestra estas opiniones en la página.
    \item \textbf{Feed de redes sociales}: Integra un feed en tiempo real de las publicaciones de redes sociales de la casa rural.
\end{itemize}

\section{Planificación y estimación de costes}
En este apartado se plantea la planificación del proyecto, abarcando el ciclo de vida a seguir, las tareas en las que se descompondrá y el coste temporal asociado.

Dado el carácter individual del proyecto, se ha optado por una metodología ágil adaptada~\cite{ilieva2004analyses}, basada en \glspl{sprint} cortos. Aunque las metodologías ágiles suelen estar diseñadas para equipos, su capacidad para priorizar tareas y validar requisitos de forma iterativa se ajusta a las necesidades de este desarrollo. La periodicidad de las revisiones permite garantizar un progreso continuo y la incorporación de mejoras basadas en las necesidades específicas del proyecto.

Para la estimación del tiempo necesario para cada tarea, se ha utilizado la técnica \gls{PERT}~\cite{mazlum2015cpm} basada en la \gls{distribucionbeta}. Este enfoque permite calcular una estimación ponderada considerando tiempos optimistas, pesimistas y el más probable. Se ha optado por la \gls{distribucionbeta} frente a otras alternativas, como la clásica o la triangular, debido a su capacidad para dar un mayor peso al tiempo más probable, lo cual es especialmente útil en proyectos donde las tareas presentan incertidumbres moderadas pero es posible realizar una estimación razonable de su duración central. Este método asegura un equilibrio entre escenarios extremos (optimista y pesimista) y la realidad del desarrollo, permitiendo obtener una planificación más ajustada.

Además, este enfoque resulta particularmente adecuado al tratarse de un único recurso, donde métodos como Planning Poker~\cite{planningpoker} o consultas a expertos no tienen aplicabilidad directa. Al ser una aplicación web personalizada con múltiples integraciones, no es un proyecto repetitivo que permita apoyarse en datos históricos, lo que refuerza la idoneidad del enfoque probabilístico de \gls{PERT}.

La planificación seguirá un enfoque secuencial, con actividades organizadas en función de sus dependencias, aunque algunas podrán ejecutarse en paralelo para optimizar el tiempo de desarrollo. Las tareas se agrupan en fases como análisis, diseño, implementación y pruebas, y se estima su duración en horas. Estas estimaciones se convertirán posteriormente a días para facilitar la planificación general.
Este desglose permitirá evaluar el esfuerzo necesario para completar el proyecto y garantizar que se cumplan los plazos establecidos.
La estimación de tiempos se calcula tal y como define la Ecuación~\ref{eq:estimacion_tiempos}.
\begin{equation}
    \text{Estimación} = \frac{O + P + (4 \cdot M)}{6}
    \label{eq:estimacion_tiempos}
\end{equation}

\noindent
\textit{Fórmula para la estimación de tiempos basada en la técnica \gls{PERT}, donde $O$ es el tiempo optimista, $P$ el pesimista y $M$ el más probable.}

Considerando la metodología ágil adoptada, la Tabla~\ref{tbl:estimacion} detalla las tareas a realizar, desglosadas por fase y con sus correspondientes estimaciones de tiempo, calculadas mediante la metodología \gls{PERT}. Además, cada tarea se ha asignado a uno de los siete sprints planificados, reflejando la organización incremental e iterativa del trabajo. De este modo, se puede visualizar no solo el esfuerzo estimado en horas, sino también la distribución de las tareas a lo largo de los diferentes ciclos de desarrollo.
\begin{table}[h!tb]
    \centering
    \begin{adjustbox}{width=0.98\textwidth}
    \begin{scriptsize}
    \begin{tabular}{|c|c|L|c|c|c|c|}
    \hline
    \textbf{Sprint} & \textbf{Fase} & \textbf{Descripción} & \textbf{T.Optimista (h)} & \textbf{T.Pesimista (h)} & \textbf{T.Mejor (h)} & \textbf{Estimación (h)} \\
    \hline
    \multirow{7}{*}{\raisebox{-10pt}{\textbf{{\Large 1}}}} & Documentación & Estado del arte & 8 & 12 & 10 & 10 \\ \cline{2-7}
    & Análisis & Requisitos & 8 & 12 & 9 & 9.33 \\ \cline{2-7}
    & Análisis & Especificación técnica & 8 & 12 & 9 & 9.33 \\ \cline{2-7}
    & Análisis & Diagramas y casos de uso & 8 & 10 & 9 & 9.33 \\ \cline{2-7}
    & Diseño & Diseño del usuario & 8 & 10 & 9 & 9.33 \\ \cline{2-7}
    & Análisis & Compatibilidad de tecnologías & 8 & 11 & 9 & 9.33 \\ \cline{2-7}
    & Diseño & Infraestructura Backend & 8 & 12 & 10 & 10 \\ \hline

    \multirow{7}{*}{\raisebox{-10pt}{\textbf{{\Large 2}}}} & Implementación & Front End: Menú principal & 8 & 10 & 9 & 9.33 \\ \cline{2-7}
    & Implementación & Formulario de contacto & 2 & 4 & 3 & 3 \\ \cline{2-7}
    & Implementación & Autenticación y autorización & 8 & 12 & 9 & 9.33 \\ \cline{2-7}
    & Implementación & Backend: Lógica y base de datos reseñas & 16 & 22 & 18 & 18.33 \\ \cline{2-7}
    & Implementación & Base de datos de usuarios y roles & 2 & 4 & 3 & 3 \\ \cline{2-7}
    & Pruebas & Unitarias: obtención de reseñas & 2 & 3 & 2.5 & 2.5 \\ \hline

    \multirow{7}{*}{\raisebox{-10pt}{\textbf{{\Large 3}}}} & Implementación & Microservicio: contenido de redes sociales & 8 & 12 & 10 & 10 \\ \cline{2-7}
    & Implementación & Recomendaciones y eventos & 3 & 5 & 4 & 4 \\ \cline{2-7}
    & Implementación & Información turística & 3 & 5 & 4 & 4 \\ \cline{2-7}
    & Implementación & Integración datos: menú principal & 8 & 12 & 10 & 10 \\ \cline{2-7}
    & Implementación & Integración datos: información turística & 2 & 4 & 3 & 3 \\ \cline{2-7}
    & Pruebas & Datos: menú principal & 2 & 3 & 2.5 & 2.5 \\ \cline{2-7}
    & Pruebas & Datos: información turística & 2 & 3 & 2.5 & 2.5 \\ \hline

    \multirow{7}{*}{\raisebox{-10pt}{\textbf{{\Large 4}}}} & Implementación & Calendario de disponibilidad & 6 & 10 & 8 & 8 \\ \cline{2-7}
    & Implementación & Sistema de reservas: Front End & 5 & 8 & 6.5 & 6.5 \\ \cline{2-7}
    & Implementación & Base de datos de reservas & 2 & 4 & 3 & 3 \\ \cline{2-7}
    & Implementación & Sistema de reservas: Backend & 10 & 15 & 12 & 12.33 \\ \cline{2-7}
    & Pruebas & Unitarias: sistema de reservas & 3 & 5 & 4 & 4 \\ \cline{2-7}
    & Implementación & Mapas interactivos & 1 & 2 & 1.5 & 1.5 \\ \cline{2-7}
    & Implementación & Pronóstico del clima & 1 & 2 & 1.5 & 1.5 \\ \hline

    \multirow{6}{*}{\raisebox{-10pt}{\textbf{{\Large 5}}}} & Implementación & Microservicio: clima API & 6 & 10 & 8 & 8 \\ \cline{2-7}
    & Implementación & Microservicio: ayuntamientos API & 12 & 18 & 14 & 14.33 \\ \cline{2-7}
    & Implementación & Actividades & 2 & 4 & 3 & 3 \\ \cline{2-7}
    & Implementación & Microservicio: actividades API & 8 & 12 & 10 & 10 \\ \cline{2-7}
    & Pruebas & Datos: actividades al aire libre & 1 & 2 & 1.5 & 1.5 \\ \cline{2-7}
    & Pruebas & Datos: recomendaciones y eventos & 1 & 2 & 1.5 & 1.5 \\ \hline

    \multirow{4}{*}{\raisebox{-10pt}{\textbf{{\Large 6}}}} & Implementación & Backend: Feed redes sociales & 4 & 7 & 5.5 & 5.5 \\ \cline{2-7}
    & Pruebas & Usuario final & 8 & 12 & 10 & 10 \\ \cline{2-7}
    & Pruebas & Seguridad & 2 & 3 & 2.5 & 2.5 \\ \cline{2-7}
    & Dockerización & Backend & 16 & 22 & 18 & 18.33 \\ \hline

    \multirow{1}{*}{\raisebox{-10pt}{\textbf{{\Large 7}}}} & Producción & Evaluación soluciones y despliegue & 8 & 12 & 10 & 10 \\ \hline

    \textbf{} & \textbf{Total} &  & \textbf{225} & \textbf{322} & \textbf{264.5} & \textbf{263.92} \\ \hline
    \end{tabular}
    \end{scriptsize}
    \end{adjustbox}
    \caption{Estimación de tareas del proyecto con asignación por sprint según planificación.}
    \label{tbl:estimacion}
\end{table}




El ciclo de vida estimado para el proyecto es de aproximadamente \textbf{14 semanas}, lo cual corresponde a un total de \textbf{263.92 horas} de trabajo. Debido a la combinación de actividades laborales y estudios, se estima una dedicación de \textbf{22 horas semanales}, distribuidas en \textbf{2 horas diarias entre semana} y \textbf{6 horas los fines de semana}, tal y como se muestra en la Tabla~\ref{tbl:horario}.

\begin{table}[h!tb]
    \centering
    \begin{adjustbox}{width=0.7\textwidth}
    \begin{tabular}{|l|l|l|l|l|l|l|}
        \hline 
        \textbf{Lunes} & \textbf{Martes} & \textbf{Miércoles} & \textbf{Jueves} & \textbf{Viernes} & \textbf{Sábado} & \textbf{Domingo} \\
        \hline 
        2h & 2h & 2h & 2h & 2h & 6h & 6h \\
        \hline
    \end{tabular}
\end{adjustbox}
    \caption{Horario de dedicación}
    \label{tbl:horario}
\end{table}


Seguidamente, en la Figura~\ref{fig:diagramaGantIncremental} se encuentra el diagrama de \gls{Gantt} generado con el paquete \texttt{Pgfgantt}~\cite{pgfgant:website}, que ilustra gráficamente la planificación temporal del proyecto. En él, cada sprint corresponde a un ciclo de trabajo de dos semanas (14 días), lo que permite observar claramente la secuencia y solapamiento de tareas en el tiempo, así como el avance progresivo del proyecto conforme a los principios de la gestión ágil.

\begin{sidewaysfigure}
    \begin{minipage}{1\linewidth}
        \centering
        \begin{adjustbox}{width=0.9\textwidth} 
            \def\pgfcalendarmonthname#1{%
  \ifcase#1
    \or Ene\or Feb\or Mar\or Abr\or May\or Jun\or Jul\or Ago\or Sep\or Oct\or Nov\or Dic\fi%
}




\def\pgfcalendarweekdayletter#1{%
    \ifcase#1
    \textcolor{White!80!black}{\tiny L}
    \or \textcolor{White!80!black}{\tiny M}
    \or \textcolor{White!80!black}{\tiny W}
    \or \textcolor{White!80!black}{\tiny J}
    \or \textcolor{White!80!black}{\tiny V}
    \or \textcolor{Yellow!80!black}{\tiny S}
    \or \textcolor{Yellow!80!black}{\tiny D}
    \fi%
}

\begin{ganttchart}[
    hgrid,
    vgrid,
    x unit=.4cm,
    y unit chart=0.8cm, 
    y unit title=0.8cm,
    time slot format=isodate,
    title/.append style={draw=none, fill=RoyalBlue!50!black},
    title label font=\sffamily\kern0.3em\color{White!90!black},
    title height=1,
    bar/.append style={draw=none, fill=OliveGreen!75, rounded corners=2pt},
    bar label font=\sffamily\kern0.3em\color{black},
    bar label node/.append style={xshift=-0.5cm, align=left},
    bar height=0.4,
    group/.append style={draw=none, fill=RoyalBlue!75, rounded corners=2pt},
    bar label font=\Large\sffamily\color{black},
    group label node/.append style={xshift=-0.5cm, align=left}, 
    group height=0.4,
    milestone/.append style={fill=orange},
    milestone label font=\Large\sffamily\bfseries\color{orange!80!black},
]{2024-11-01}{2025-02-06}

    % Títulos del calendario
    \gantttitle{Planificación del proyecto: Desde 01/11/2024 hasta 28/02/2025}{98} \\
    \gantttitle{Sprint 1}{14}
    \gantttitle{Sprint 2}{14}
    \gantttitle{Sprint 3}{14}
    \gantttitle{Sprint 4}{14}
    \gantttitle{Sprint 5}{14}
    \gantttitle{Sprint 6}{14}
    \gantttitle{Sprint 7}{14}
     \\
    \gantttitlecalendar{month=name, weekday=letter} \\
    % Tareas con fechas
    \ganttbar{Documentación: Estado del arte}{2024-11-01}{2024-11-05} \\
    \ganttbar{Análisis: Requisitos}{2024-11-06}{2024-11-07} \\
    \ganttbar{Análisis: Especificación técnica}{2024-11-08}{2024-11-12} \\
    \ganttbar{Análisis: Diagramas y casos de uso}{2024-11-13}{2024-11-14} \\
       \ganttmilestone{Fin Sprint 1}{2024-11-14} \\
    \ganttbar{Diseño: Diseño del usuario}{2024-11-15}{2024-11-20} \\
    \ganttbar{Análisis:  Evaluación de compatibilidad tecnológica}{2024-11-21}{2024-11-22} \\
    \ganttbar{Diseño: Infraestructura Backend}{2024-11-22}{2024-11-26} \\
    \ganttbar{Implementación: Front End: Menú principal}{2024-11-27}{2024-11-28} \\
    \ganttmilestone{Fin Sprint 2}{2024-11-28} \\
    \ganttbar{Implementación: Formulario de contacto}{2024-11-29}{2024-11-29} \\
    \ganttbar{Implementación: Autenticación y autorización}{2024-11-29}{2024-12-02} \\
    \ganttbar{Implementación: Backend: Lógica y base de datos reseñas}{2024-12-03}{2024-12-08} \\
    \ganttbar{Implementación:  Base de datos de usuarios y roles}{2024-12-08}{2024-12-08} \\
    \ganttbar{Pruebas Unitarias: obtención de reseñas}{2024-12-09}{2024-12-09} \\
    \ganttbar{Implementación: Microservicio: contenido de redes sociales}{2024-12-10}{2024-12-12} \\
    \ganttmilestone{Fin Sprint 3}{2024-12-12} \\
    \ganttbar{Implementación: Recomendaciones y eventos}{2024-12-13}{2024-12-14} \\
    \ganttbar{Implementación: Información turística}{2024-12-15}{2024-12-16} \\
    \ganttbar{Implementación: Integración datos: menú principal}{2024-12-17}{2024-12-18} \\
    \ganttbar{Implementación: Integración datos: información turística}{2024-12-19}{2024-12-19} \\
    \ganttbar{Pruebas: Integración de Datos: menú principal}{2024-12-20}{2024-12-20} \\
    \ganttbar{Pruebas:  Integración de Datos: información turística}{2024-12-21}{2024-12-21} \\
    \ganttbar{Implementación: Calendario de disponibilidad}{2024-12-22}{2024-12-24} \\
    \ganttbar{Implementación: Sistema de reservas: Front End}{2024-12-24}{2024-12-25} \\
    \ganttbar{Implementación: Base de datos de reservas}{2024-12-26}{2024-12-26} \\
    \ganttmilestone{Fin Sprint 4}{2024-12-26} \\
    \ganttbar{Implementación: Sistema de reservas: Backend}{2024-12-27}{2024-12-30} \\
    \ganttbar{Pruebas Unitarias: sistema de reservas}{2024-12-31}{2024-12-31} \\
    \ganttbar{Implementación: Mapas interactivos}{2024-12-31}{2024-12-31} \\
    \ganttbar{Implementación: Pronóstico del clima}{2025-01-01}{2025-01-01} \\
    \ganttbar{Implementación: Microservicio: clima API}{2025-01-02}{2025-01-04} \\
    \ganttbar{Implementación: Microservicio: ayuntamientos API}{2025-01-05}{2025-01-07} \\
    \ganttbar{Implementación: Actividades}{2025-01-08}{2025-01-09} \\
    \ganttmilestone{Fin Sprint 5}{2025-01-09} \\
    \ganttbar{Implementación: Microservicio: actividades API}{2025-01-10}{2025-01-14} \\
    \ganttbar{Pruebas Integración de Datos:  actividades al aire libre}{2025-01-14}{2025-01-15}\\
    \ganttbar{Pruebas Integración de Datos: recomendaciones y eventos}{2025-01-16}{2025-01-16} \\
    \ganttbar{Implementación: Backend: Feed redes sociales}{2025-01-16}{2025-01-17} \\
    \ganttbar{Pruebas: Usuario final}{2025-01-18}{2025-01-20} \\
    \ganttbar{Pruebas: Seguridad}{2025-01-21}{2025-01-23} \\
    \ganttmilestone{Fin Sprint 6}{2025-01-23} \\
    \ganttbar{Dockerización: Backend y microservicios}{2025-01-24}{2025-01-29} \\
    \ganttbar{Evaluación y despliegue nube}{2025-01-30}{2025-02-06}\\
\ganttmilestone{Fin Sprint 7}{2025-02-06} \\
\end{ganttchart}
 
        \end{adjustbox}
        \caption{Diagrama de Gantt del proyecto}
        \label{fig:diagramaGantIncremental}
    \end{minipage}
\end{sidewaysfigure}


A continuación, se va a estimar el coste de personal del proyecto. Para este proyecto se ha destinado 1 persona durante 291 horas (264 estimadas y un 10\% margen). Basándome en el estudio Michael Page de salarios \cite{salarios:2024} para un perfil de desarrollador Back End y Front End en Valencia con menos de dos años de experiencia el salario bruto esta entre los 22.000\euro\space y los 40.000\euro\space por lo tanto se va a coger la media de estas dos cifras 31.000\euro\space al año que convertido a salario mensual da un total de 2.583,40\euro\space por mes. Por lo tanto, el salario por hora contando un trabajo a jornada completa de 40 horas es de 16,15\euro/h\space. Multiplicado por el número de horas estimadas, da un total de 4.699,95\euro\space al cual se le debe sumar la correspondiente cotización en la seguridad social \cite{cotizacion:web} que en este caso es del 28,3\%. Por lo tanto, se obtendría un coste del proyecto de 6.015,94 \euro.

Otro aspecto a tener en cuenta son los costes materiales y la amortización que se va a hacer de ellos durante el desarrollo del proyecto. Lo primero que se va a calcular es la amortización de los equipos informáticos a utilizar en el desarrollo del proyecto. El porcentaje de amortización vigente para sistemas informáticos según la \gls{aeat} \cite{amortizacion:2024} es del 26\% y su vida útil estará entre 4 y 8 años por lo que se selecciona una vida útil para todos los componentes de 4 años (48 meses) para una mayor amortización, aplicando la Fórmula \ref{eq:amortiza}.
\begin{equation}
\label{eq:amortiza}
    \frac{Coste\;del\;material}{4}*A\tilde{n}os\;amortizaci\acute{o}n\;(uso\;material)
\end{equation}
La Tabla~\ref{tbl:amortizacion} muestra los costes a imputar de cada uno de los componentes utilizados. En este caso sólo se dispone de un componente ya que los demás componentes forman parte de la infraestructura del proyecto y pasan a ser costes directos. Ya que el único componente de coste directo sería la utilización de un entorno de plataforma en la nube \gls{paas} y que el coste de este variará en función de su utilización, no es posible añadirlo a la estimación.


\begin{table}[h!tb]
		
		\begin{tabular}{|l|l|c|c|}
			\hline
			\textbf{Elemento} & \textbf{Coste} & \textbf{\begin{tabular}[c]{@{}c@{}}Meses \\ de \\ utilización\end{tabular}} & \textbf{\begin{tabular}[c]{@{}c@{}}Coste\\de\\Amortización\\ \end{tabular}} \\ \hline
			\multicolumn{1}{|c|}{\begin{tabular}[c]{@{}c@{}}Ordenador \\ para desarrollo\end{tabular}} & \multicolumn{1}{c|}{890,00 \euro} & 4                                                                           &   74,16 \euro                                                  \\ \hline
		\end{tabular}
	 \centering
	\caption{Estimación de la amortización de materiales}
	\label{tbl:amortizacion}
\end{table}


Para finalizar, la Tabla~\ref{tbl:costes} muestra el coste de cada activo a utilizar durante el proyecto, incluyendo materiales, equipos de desarrollo y personal:
\begin{table}[h!tb]
\begin{adjustbox}{width=0.5\textwidth}
		\begin{tabular}{|c|c|}
			\hline \textbf{Elemento}
			& \textbf{Coste (\euro)} \\
			
			\hline Amortización portátil para desarrollo
			& 74,16  \\
			\hline Recursos humanos
			& 6.015,94 \\
			\hline \textbf{Total}
			& \textbf{6.090,11 \euro} \\
			\hline
		\end{tabular}
	\end{adjustbox}
	 \centering
	\caption{Estimación costes del proyecto}
	\label{tbl:costes}
\end{table}


\section{Riesgos}
En el transcurso del desarrollo existe una gran cantidad de módulos a implementar que posteriormente deben encajar y esto es posiblemente lo más costoso del proyecto y por lo que se va a dedicar tanto tiempo al desarrollo del sistema principal y a implantarlo con los demás subsistemas. Esta planificación por tanto lleva asociados unos riesgos que podrían retrasar el ciclo de vida del proyecto, los cuales se enumerarán a continuación:
\begin{enumerate}
	
	\item Falta de tiempo por actividad laboral.
	\item Problemas de compatibilidad entre las \glspl{API} a utilizar.
	\item Problemas de compatibilidad con plataformas de despliegue en la nube utilizadas.
	
\end{enumerate}

Una vez analizados los posibles riesgos detectados, se plantea el plan de contingencia de la Tabla~\ref{tbl:contingencia}.

\begin{table}[h!tb]
    \centering
    \begin{tabular}{|p{0.25\textwidth}|p{0.2\textwidth}|p{0.45\textwidth}|}
        \hline 
        \textbf{Riesgo} & \textbf{Coste} & \textbf{Solución} \\
        \hline 
        Falta de tiempo por actividad laboral. & 
        2 horas por semana & 
        Incrementar el tiempo dedicado el fin de semana de 6 a 8 horas para compensar días con mayor carga de trabajo entre semana \\
        \hline 
         Problemas de compatibilidad entre las \glspl{API} a utilizar & 1 semana & A la hora de integrar distintas \glspl{API}, pueden surgir incompatibilidades por cambios en sus versiones, diferencias en el formato de los datos o limitaciones de acceso. Este tipo de problemas podría suponer un retraso considerable, especialmente si afecta a funcionalidades clave. Para reducir el impacto, he analizado previamente varias alternativas para cada \gls{API} principal. Así, si surge algún inconveniente grave, podré cambiar rápidamente a otra opción sin necesidad de rehacer gran parte del trabajo. También realizaré pruebas de integración desde las primeras fases del desarrollo para detectar posibles incompatibilidades lo antes posible. \\ & 
      \\
        \hline 
        Problemas de compatibilidad con plataformas de despliegue en la nube utilizadas. & 
        1 semana & 
        Requisito opcional: El uso de servicios en la nube puede provocar incidencias relacionadas con la configuración de la infraestructura, incompatibilidades entre entornos o restricciones en los servicios contratados. Si llegara a ocurrir algún problema que impida el despliegue en la nube, el proyecto podría retrasarse o no completarse según lo previsto. Por eso, tengo previsto como solución de contingencia realizar el despliegue final en un entorno local (en mi propio equipo o red local), asegurando que la funcionalidad del sistema pueda ser demostrada aunque el entorno cloud no esté disponible a tiempo. Durante el desarrollo, procuraré mantener siempre operativa esta alternativa local como respaldo. \\
        \hline
    \end{tabular}

    \caption{Plan de contingencias para los riesgos detectados}
    \label{tbl:contingencia}
\end{table}

Además de las soluciones planteadas anteriormente para cada riesgo detectado, en la planificación del proyecto se ha tenido en cuenta un margen de error del 10\% (1 último sprint y una semana respecto a lo estimado que equivale a 27 horas) de margen para poder solucionar los riesgos valorados anteriormente.

\section{Viabilidad}

Teniendo en cuenta los puntos observados anteriormente se puede decir que el proyecto está planificado para su entrega en el mes de marzo de 2025. Empezando el desarrollo en el mes de noviembre de 2024 y finalizándolo durante el mes de febrero de 2025. En cuanto a su coste, se ha estimado \textbf{6.090,11 \euro}. Es un coste medio por lo que en caso del desarrollo de un proyecto real con esta estimación no habría problema a la hora de asumirlo.\\
Su coste temporal es de 98 días teniendo como margen 21 días más, por lo que podría alargarse el proyecto hasta principios de marzo. Es un proyecto que podría realizarse en paralelo si se tuviesen varios desarrolladores, en concreto uno encargado del \gls{frontend} y otro para \gls{backend}. Pero en este caso se ha optado por un desarrollo secuencial. Dado que las tecnologías a utilizar son conocidas sería bastante realista al decir que la planificación estimada se va a cumplir dentro de los márgenes establecidos.\\
